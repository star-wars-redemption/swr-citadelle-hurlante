\section{\'Epilogue}

Une fois la \nameref{sec:ktath-atn-queen} éliminée, les héros ne risquent plus rien dans la citadelle. Tous les gardes sont redevenus eux même et les héros ont sauvé les habitants de la planète.

Faites faire à tous les joueurs un jet de \textbf{Per}ception. Ceux qui réussissent s’aperçoivent que \nameref{sec:aphra} est en train de discrètement se faire la malle. C’est le moment de régler leur comptes. Si personne ne réussit son jet, \nameref{sec:aphra} s’en va et personne ne s’en aperçoit à temps.
Du coup laissez les héros entamer la discussion.

\begin{paperbox}{Aphra}
Selon les questions qu’ils posent et comment évolue la conversation. Voilà ce qu’il faut savoir :

\begin{rebelist}
    \item \nameref{sec:aphra} est bien la trafiquante de droïde qu’ils recherchent. Mais elle ne l’avouera jamais, elle tentera plutôt de les amener vers une autre piste.
    \item L’artefact, l’\textbf{Ordu Aspectu} contient la personnalité d’un Jedi nommé \textbf{Rur}, mais ce dernier est devenu fou depuis qu’il a été enfermé dans ce cristal. On ne peut pas en faire grand-chose.
    \item Pour se défausser, elle expliquera que son objectif a toujours été de sauver les habitants de \textit{Ktath'Atn} mais qu’elle avait peur que vous refusiez. En réalité, quand elle a compris que le cristal ne pourrait pas lui servir à grand-chose elle a eu des scrupules de laisser les héros à leur sort pour rien.
\end{rebelist}
\end{paperbox}

Les héros peuvent alors décider du sort d’\nameref{sec:aphra}, la capturer et l’amener à leur supérieur (Empire ou Alliance), ou la laissée repartir en promettant de ne plus vendre de droïdes à la faction opposée. Dans tous les cas, \nameref{sec:aphra} est prise au piège, elle ne se défendra pas.

\begin{paperbox}{Récompenses}
Du coup, c’est un scénario assez long au final avec pas mal de combats difficiles. Prévoir \textbf{3 XP} comme salaire de base.
\begin{rebelist}
    \item \textbf{+1 XP} Pour les héros qui ont eu des informations durant la réception
    \item \textbf{+1 XP} S’ils ont compris qu’\nameref{sec:aphra} est leur trafiquante de matrices de droïdes.
\end{rebelist}
\end{paperbox}