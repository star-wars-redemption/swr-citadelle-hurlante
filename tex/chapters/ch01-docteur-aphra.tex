\section{Docteur Aphra}

\subsection{introduction}
Ce scénario commence dans une cantina pas vraiment accueillante d’Horox III. Les héros aspirent à un peu de repos bien mérité après une mission pour l’alliance qui n’a pas vraiment été un franc succès.

\lettrine{\jedifont{\$}} Ils avaient pour objectif d’identifier l’origine d’un trafic de matrices droïde modifiées pour le combat. Malheureusement après deux semaines de recherches et d’interrogatoires infructueux, tout ce qu’ils ont réussi à faire, c’est se prendre une correction par l’un des fameux droïdes modifiés et perdre la trace de leur trafiquant quelque part sur \textbf{Horox III}.

\lettrine{\jedifont{\#}} Les héros étaient en mission pour l’empire. Ils sont à la recherche d’un individu capable de modifier les matrices de droïde pour en faire des droïdes de combat plutôt performant. Le seigneur Vador se montre très intéressé car il souhaite se monter une armée personnelle.

\begin{paperbox}{Objectif}
Les héros sont à la recherche d’un trafiquant de matrices de droïdes.
\end{paperbox}

\subsection{Viens voir le docteur}
\noindent\includegraphics[width=\linewidth]{_img/places/cantina-horox-iii.jpg}
Les voilà donc ruminant leur échec dans un bar quand entre une femme, brune, mince, un bonnet et des lunettes d’aviateur sur la tête. L’horoxien au bar lève la tête et semble, l’espace d’un instant, surpris. Puis il se reprend et interpelle violemment la femme :

\begin{quotebox}
- \textbf{barman} Aphra ! Tu ne manques pas d’aplomb de te repointer ici ! Les droïdes que tu m’a refilés, c’était de la merde. Ils ont pas tenu 10 minutes !\\
- \textbf{Aphra} En face de ta sale gueule ça m’étonne pas !\\
- \textbf{barman} Attrapez là, on va lui faire sa fête !
\end{quotebox}

Les \nameref{sec:horoxian-barfly}, jusqu’alors calmes, occupés à leurs consommations se lèvent et commencent à pousser les tables. L’ambiance devient très tendue, la bagarre est inévitable.

Si vos joueurs ont un peu de jugeote, ils comprennent qu’\nameref{sec:aphra} a quelque chose à voir avec les matrices modifiées qu’ils recherchent, ou qu’au moins elle peut les aider. S’ils ne comprennent pas incitez les à entrer dans la danse avec une bouteille perdue par exemple.

\begin{paperbox}{Baston de bar}
Comptez 2 \nameref{sec:horoxian-barfly} par héros pour que ça soit un peu marrant. Faites aussi en sorte qu’ils n’utilisent pas leurs armes, ça serait trop facile sinon. En gros la mission était finie et ils n’ont pas pris leurs armes avec eux, ou les armes sont interdites dans la cantina.

Au pire, s’ils s’en sortent pas, les \nameref{sec:horoxian-barfly} vont fuir.
\end{paperbox}

\subsection{Docteur \& Queen}

Une fois la baston terminée, les héros interpellent \nameref{sec:aphra}, s’ils n’en font rien, c’est elle qui les accoste.

Soit ils lui demandent des infos sur les matrices de droïdes et elle leur propose un marché, soit, elle leur demande de l’aide pour résoudre un problème en échange d’argent ou d’un artefact Jedi. Tout dépend ce qui fait rêver vos joueurs.

\nameref{sec:aphra} leur explique alors son problème. Elle possède un artefact, le cristal de Rur, contenant la mémoire d’un Jedi mais elle ignore comment l’ouvrir. Par chance, une fois par an, la \nameref{sec:ktath-atn-queen} organise une réception durant laquelle elle accorde un vœu à la personne qui lui apportera la forme de vie organique la plus “étonnante”. Et il se trouve que cette réception a justement lieu demain soir, et présenter un spécimen d’humain sensible à la Force par les temps qui courent leur assurerait à coup sûr l’accès au vœu. S’il se pose la question, l’artefact réagit aux êtres sensibles à la Force.

\bigbreak

Une fois la discussion terminée, laissez les héros récupérer leurs armes et faire le plein de munition puis en route pour \textbf{Ktath’Atn.}

\begin{paperbox}{Objectif}
Aider \nameref{sec:aphra} à ouvrir l’artefact pour obtenir les renseignements sur le trafic de matrices.
\end{paperbox}
