\section{nuit hurlante}

Les héros vont donc se coucher avec le sentiment d’une mission accompli, pour une reposante nuit de sommeil. Ou pas ...

\begin{paperbox}{Ouvrez l’œil}
Si certains de vos héros ont choisit de ne pas dormir pour monter la garde ou parce qu’ils s’attendent à un piège, ils ont le l’occasion de faire un jet de \textbf{Per}ception. Ceux qui le réussissent \textbf{entendent des cliqueties sur le sol, comme des bruits d’insecte}. Ils ont alors 10s (temps réel) pour faire une action.
\end{paperbox}

Débarque alors dans la chambre X (mettre ici un ou deux de plus que le nombre de vos héros + Aphra) bestioles. Des sortes d’araignés avec une carapace externe, un peu plus grosse que le poing et visiblement pas là pour faire des massages aux invités. Si aucun de vos héros n’avais jugé bon de garder un oeil ouvert, ils n’ont pas l’avantage et vont devoir se dépétrer des bestioles avant toute autre action. Pour cela, chaque héro "non averti" fait un jet de \textbf{Par}ade, pour se dégager de la bestiole qui lui a sauté dessus. Chaque échec lui coute un point de fatigue. Un héro libre peu venir préter main forte a son partenaire.

Les bestioles ne subissent aucun dégât et reviennent sans cesse à l’assault. Les héros doivent fuir se mettre à l’abrit. S’ils ne font rien dans ce sens, \nameref{sec:aphra} le leure fait comprendre. Il faut quitter cet endroit au plus vite.