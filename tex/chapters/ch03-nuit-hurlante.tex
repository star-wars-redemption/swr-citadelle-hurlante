\section{nuit hurlante}

\subsection{Réveil mouvementé}

Les héros vont donc se coucher avec le sentiment d’une mission accompli, pour une reposante nuit de sommeil. Ou pas ...

\begin{paperbox}{Ouvrez l’œil}
Si certains de vos héros ont choisit de ne pas dormir pour monter la garde ou parce qu’ils s’attendent à un piège, ils ont le l’occasion de faire un jet de \textbf{Per}ception. Ceux qui le réussissent \textbf{entendent des cliqueties sur le sol, comme des bruits d’insecte}. Ils ont alors 10s (temps réel) pour faire une action.
\end{paperbox}

Débarque alors dans la chambre X (mettre ici un ou deux de plus que le nombre de vos héros + Aphra) bestioles. Des sortes d’araignés avec une carapace externe, un peu plus grosse que le poing et visiblement pas là pour faire des massages aux invités. Si aucun de vos héros n’avais jugé bon de garder un oeil ouvert, ils n’ont pas l’avantage et vont devoir se dépétrer des bestioles avant toute autre action. Pour cela, chaque héro "non averti" fait un jet de \textbf{Par}ade, pour se dégager de la bestiole qui lui a sauté dessus. Chaque échec lui coute un point de fatigue. Un héro libre peu venir préter main forte a son partenaire.

Les bestioles ne subissent aucun dégât et reviennent sans cesse à l’assault. Les héros doivent fuir se mettre à l’abrit. S’ils ne font rien dans ce sens, \nameref{sec:aphra} le leur fait comprendre. Il faut quitter cet endroit au plus vite.

\nameref{sec:aphra} appelle à la rescousse ses deux droïdes :

\begin{quotebox}
\noindent\textbf{\nameref{sec:varroa}}: Triple 0, BT-1 maniez vous de venir nous récupérer !
\noindent\textbf{0-0-0}: On est un peu occupé là !
\noindent\textbf{\nameref{sec:varroa}}: Maniez vous ! Ou the vous juge que je vous démonte !
\end{quotebox}

Quelques minute plus tard tard le vaisseau vient s’écraser lamentablement devant nos héros, ouvrant une brèche dans l’enceinte. 

\begin{quotebox}
\noindent\textbf{\nameref{sec:varroa}}: Putain mais c’est trop compliqué de piloter un vaisseau sans se crouter !
\noindent\textbf{0-0-0}: Je vous ai dis qu’on était occupé ! Occupé à aligner le propulseur de poupe !
\end{quotebox}

Le temps que les retrouvailles se terminent, voilà que débarque \nameref{sec:bombinax}, l’un des trois sbires de la reine. La bonne grosse brute en armure lourde. Le gars est balaise et vos héros vont devoir utiliser tous l’attirail d’actions à leur disposition pour s’en débarasser (Intimidation, Sarcasme, Viser, ...).

\subsection{Symbiotes Abersyn}

Après le combat, les héros remarquent une porte entre-ouverte qui semble assé solide pour les mettre à l’abrit le temps d’échaffauder un plan. Un des droïdes peu fermer les portes blindés. Une fois les héros à l’abrit, ils peuvent observé la pièce. 

La pièce est assez grande et il semble que ce soit un laboratoire. Sur les cotés de la pièce on observe des cuves remplis d’un liquide vert, contenant des spécimens des mêmes créatures que celles qui ont attaqués les héros ce matin. Un pupitre avec une console se trouve relié aux cuves par des cables. \\

Si les héros n’ont pas de droïde, c’est BT-1 qui s’y colle et qui connecte les héros à la console. Ils y apprennent ce que sont les \nameref{sec:symbiote-abersyn}.

En plus des informations sur les \nameref{sec:symbiote-abersyn}s, la console contient un plan détaillé de la citadelle. Sont repérables dans le complexe :

\begin{rebelist}
	\item Un hangard à vaisseau
	\item Plusieurs autres laboratoires
	\item Une pièce qui semble être le centre d’une activité
    \item Les appartement de la reine
\end{rebelist}

Laissez les héros faire leur choix de ce qu’ils veulent faire ? Fuir, retrouver la reine pour ouvrir l’artéfacte, détruire le complexe, ... Enbon MJ, peut importe le choix on leur mettra des batons dans les roues de toutes façon !